\documentclass{beamer}
\usepackage{ctex, hyperref}
\usepackage[T1]{fontenc}

% other packages
\usepackage{latexsym,amsmath,xcolor,multicol,booktabs,calligra}
\usepackage{graphicx,pstricks,listings,stackengine}

\author{\kaishu{汪洋}}
\title{Introduction to OpenFOAM Programming}
\subtitle{01-初始准备}
\institute{\kaishu{武汉理工大学交通学院}}
\date{\kaishu{2020年12月}}
\usepackage{WUT}

% defs
\def\cmd#1{\texttt{\color{red}\footnotesize $\backslash$#1}}
\def\env#1{\texttt{\color{blue}\footnotesize #1}}
\definecolor{deepblue}{rgb}{0,0,0.5}
\definecolor{deepred}{rgb}{0.6,0,0}
\definecolor{deepgreen}{rgb}{0,0.5,0}
\definecolor{halfgray}{gray}{0.55}

\lstset{
    basicstyle=\ttfamily\small,
    keywordstyle=\bfseries\color{deepblue},
    emphstyle=\ttfamily\color{deepred},    % Custom highlighting style
    stringstyle=\color{deepgreen},
    numbers=left,
    numberstyle=\small\color{halfgray},
    rulesepcolor=\color{red!20!green!20!blue!20},
    frame=shadowbox,
}


\begin{document}

\kaishu
\begin{frame}
    \titlepage
    \begin{figure}[htpb]
        \begin{center}
            \includegraphics[width=0.15\linewidth]{pic/WUT.png}
        \end{center}
    \end{figure}
\end{frame}

\begin{frame}
   \tableofcontents[sectionstyle=show,subsectionstyle=show/shaded/hide,subsubsectionstyle=show/shaded/hide]
\end{frame}

\section{课程目的}

\begin{frame}{CFD中的开源软件}
    \begin{itemize}
        \item 简要介绍CFD中的开源软件
        \item 如何使用OpenFOAM
        \item 如何修改OpenFOAM
    \end{itemize}
\end{frame}

\begin{frame}{课程安排}
    \begin{itemize}
        \item 第一阶段旅程,2天
        \item 第二阶段旅程,2天
        \item 第三阶段旅程,2天
        \item 汇报
    \end{itemize}
\end{frame}

\begin{frame}{第一阶段安排}
    \begin{itemize}
        \item 介绍OpenFOAM标准求解器、工具和相关库
        \item 介绍Paraview(ParaFoam)进行后处理
        \item 介绍OpenFOAM的tutorial的相关组织
        \item 快速入门Python,Gnuplot等工具
        \item 使用utilites和functionObjects做作业
        \item 学会课后自学
    \end{itemize}
    
\end{frame}

\begin{frame}{第二、三阶段安排}
    \begin{itemize}
        \item OpenFOAM源码目录
        \item 高级编程和编译-应用程序(求解器和工具)
        \item C++快速课程和面向对象,目的为了阅读OF源码
        \item 高级编程和编译-库(湍流模型和边界条件)
        \item Debugging
        \item 学会课后自学
        \item 作业
    \end{itemize}
    
\end{frame}

\begin{frame}{课程收获}
    \begin{itemize}
        \item 学会下载、安装、编译和运行标准OpenFOAM求解器和工具
        \item 学会何实现求解器和工具
        \item 学会实现一个湍流模型
        \item 学会实现一种边界条件
        \item 学会基础的C++和面向对象
        \item 学会利用Python,Gnuplot,Paraview等工具与OpenFOAM结合进行CFD计算。
        \item 学会基础的Linux,Doxygen,Comilation,precedures,Debugging,Version control system和VTK
        \item 学会用OpenFOAM做一个项目
    \end{itemize}
    
\end{frame}

\begin{frame}{参考}
    \begin{itemize}
        \item OpenFOAM, \url{www.openfoam.com}, \url{www.openfoam.org}, \url{www.foam-extend.org}
        \item OpenFOAM Wiki: \url{www.openfoamwiki.net}
        \item OpenFOAM User Guide, Programming Guide, Doxygen
        \item OpenFOAM Forum: \url{www.cfd-online.com/Forums/openfoam/}
        \item \cite{ferziger2002computational}
        \item \cite{moukalled2016finite}
        \item \cite{versteeg2007introduction}
    \end{itemize}
    
\end{frame}


\section{参考文献}

\begin{frame}[allowframebreaks]
    \bibliography{ref}
    \bibliographystyle{alpha}
    % 如果参考文献太多的话,可以像下面这样调整字体:
    % \tiny\bibliographystyle{alpha}
\end{frame}

\begin{frame}
    \begin{center}
        {\Huge\calligra Thanks!}
    \end{center}
\end{frame}

\end{document}